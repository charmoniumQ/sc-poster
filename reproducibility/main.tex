% LaTeX template for the Supercomputing Conference series Artifact Description (AD) appendix  
% V20180327
% (C)opyright 2018

% Derived with permission by Michael Heroux (Sandia National Laboratories, St. John's University, MN)
% from ae-20160509.tex 
% written by Grigori Fursin (cTuning foundation, France and dividiti, UK) 
% and Bruce Childers (University of Pittsburgh, USA)
% (C)opyright 2014-2016

% acmart is available at https://www.acm.org/publications/proceedings-template
%\documentclass[sigconf,twocolumn]{acmart}
% IEEETrans is available at https://www.ieee.org/conferences_events/conferences/publishing/templates.html
\documentclass{IEEEtran}
\usepackage{hyperref}
\usepackage{minted}
\begin{document}

\special{papersize=8.5in,11in}

\appendices

\section{Artifact Description Appendix: ``NautDB: Towards a Hybrid Runtime for Processing Compiled Queries''}

%%%%%%%%%%%%%%%%%%%%%%%%%%%%%%%%%%%%%%%%%%%%%%%%%%%%%%%%%%%%%%%%%%%%%
\subsection{Abstract}

I have written a simple database which I then compile one version on Linux and another into Nautilus (I say \textit{into} because it is compiled inside the Nautilus Aerokernel). My experiments compares the performance of both on an AMD EPYC 7281 server.

\begin{itemize}
\item The Linux binary is ran on a server running fedora.
\item The Nautilus binary is ran on the same server, at bare-metal level. Running Nautilus in QEMU or another virtual environment can be an error-detecting step, but it will not have realistic performance.
\end{itemize}

%%%%%%%%%%%%%%%%%%%%%%%%%%%%%%%%%%%%%%%%%%%%%%%%%%%%%%%%%%%%%%%%%%%%%
\subsection{Description}

\subsubsection{Check-list (artifact meta information)}

{\em Fill in whatever is applicable with some informal keywords and remove the rest}

{\small
\begin{itemize}
  \item {\bf Program:} \verb+db-multiverse+
  \item {\bf Run-time environment:} Linux 4.17.6 (in Fedora Release 28) and Nautilus @ commit \texttt{2fb4e52816}
  \item {\bf Hardware:} AMD EPYC 7281 16-Core Processor
  \item {\bf Compiler:} gcc 8.1.1 20180712 (Red Hat 8.1.1-5).
  \item {\bf Compiler Options:} See the \texttt{Makefile} and the \texttt{nautilus/Makefile} for compile-options.
  \item {\bf Experiment workflow:}
    \begin{enumerate}
    \item Clone
    \item Compile linux version
    \item Run and capture output
    \item Compile Nautilus version
    \item Boot into Nautilus and capture output
    \item After modifications, return to step 2
    \end{enumerate}
  \item {\bf Experiment customization:} Modify what work the database is doing, the number of chunks, number of columns, chunk-size, and domain-size (domain of the elements in the database).
  \item {\bf Publicly available:} Yes
\end{itemize}
}

\subsubsection{How software can be obtained (if available)}

NautDB can be cloned from this \href{https://github.com/HExSA-Lab/db-multiverse/}{GitHub Repository}.

\subsubsection{Hardware dependencies}

At the time of this writing, Nautilus supports x86\_64, Xeon Phi, and the GEM5 simulator.

In order to gather performance data, I use the CPU-enabled performance counters. These are specific within processor families. I have targetted `AMD EPYC 7281 16-Core Processor', but the code can be modified for other processors as well. If this is not modified, you can still collect cycle-counts.

\subsubsection{Software dependencies}

\texttt{gcc, GNU make, locate, GNU libc (for Linux version)}

%%%%%%%%%%%%%%%%%%%%%%%%%%%%%%%%%%%%%%%%%%%%%%%%%%%%%%%%%%%%%%%%%%%%%
\subsection{Installation}

\begin{enumerate}
\item Download the source code.

\begin{minted}{shell}
git clone git@github.com:HExSA-Lab/db-multiverse.git
\end{minted}
\end{enumerate}

%%%%%%%%%%%%%%%%%%%%%%%%%%%%%%%%%%%%%%%%%%%%%%%%%%%%%%%%%%%%%%%%%%%%%
\subsection{Experiment workflow}

\begin{enumerate}
  \item Compile the Linux version
\begin{minted}{shell}
make main
\end{minted}

  \item Run and collect output
\begin{minted}{shell}
./main > linux_output
\end{minted}

  \item Compile Nautilus
\begin{minted}{shell}
./scripts/insert_into_nautilus.sh
make -c nautilus nautilus.bin
\end{minted}

  \item Boot into Nautilus. If you are using grub, 
\begin{minted}{shell}
mv nautilus.bin /boot/nautilus.bin
echo <<EOF
menuentry "Nautilus" {
    # adjust this for your specific hardware
    set root='hd0,msdos1'
    multiboot2 /nautilus.bin
    boot
}
EOF >> /etc/grub.d/41_custom
reboot
# wait for grub menu
# select Nautilus from the menu
# capture output over serial link
\end{minted}

This is the simplest way of reproducing the Nautilus experiment. However, I recommend using a hardware management tool such as `IPMI' to automate the reboting process. Then you can reboot, select from the Grub menu, and capture the serial remotely. The other scripts in \texttt{./scripts/*} do exactly this. In this case, Expect can be used to automate navigating the Grub menu.

\end{enumerate}

%%%%%%%%%%%%%%%%%%%%%%%%%%%%%%%%%%%%%%%%%%%%%%%%%%%%%%%%%%%%%%%%%%%%%
\subsection{Evaluation and expected result}

The software will output blocks of CSV data wrapped in curly-braces, such as:

\begin{minted}{text}
file: cool_data.csv {
x column1 name,column2 name,title
1,2,
}
\end{minted}

The independent variables have a header beginning with `\texttt{x}'.

%%%%%%%%%%%%%%%%%%%%%%%%%%%%%%%%%%%%%%%%%%%%%%%%%%%%%%%%%%%%%%%%%%%%%
\subsection{Experiment customization}

Customize \texttt{src/app/main.c} to choose which modules to run. To customize the \texttt{test\_db} module, edit \texttt{src/app/test\_db.c}. The parameters for the database (number of columns, chunk size, number of chunks, domain size) and which operators will be timed can be customized here.

\end{document}

%%% Local Variables:
%%% mode: latex
%%% TeX-master: t
%%% End:
