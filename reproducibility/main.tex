% LaTeX template for the Supercomputing Conference series Artifact Description (AD) appendix  
% V20180327
% (C)opyright 2018

% Derived with permission by Michael Heroux (Sandia National Laboratories, St. John's University, MN)
% from ae-20160509.tex 
% written by Grigori Fursin (cTuning foundation, France and dividiti, UK) 
% and Bruce Childers (University of Pittsburgh, USA)
% (C)opyright 2014-2016

% acmart is available at https://www.acm.org/publications/proceedings-template
% \documentclass[sigconf,twocolumn]{acmart}
% IEEETrans is available at https://www.ieee.org/conferences_events/conferences/publishing/templates.html
\documentclass{IEEEtran}
\usepackage{hyperref}
\usepackage{minted}
\usepackage{enumitem}
\begin{document}
\newminted{shell}{frame=single,autogobble=true}
\special{papersize=8.5in,11in}

\appendices

\section{Artifact Description Appendix: ``NautDB: Towards a Hybrid Runtime for Processing Compiled Queries''}

%%%%%%%%%%%%%%%%%%%%%%%%%%%%%%%%%%%%%%%%%%%%%%%%%%%%%%%%%%%%%%%%%%%%% 
\subsection{Abstract}

I have written a simple database which I then compile one version for Linux and another for Nautilus.

\end{itemize}

%%%%%%%%%%%%%%%%%%%%%%%%%%%%%%%%%%%%%%%%%%%%%%%%%%%%%%%%%%%%%%%%%%%%% 
\subsection{Description}

\subsubsection{Check-list (artifact meta information)}

{\small
  \begin{itemize}
  \item {\bf Program:} \verb+db-multiverse+
  \item {\bf Run-time environment:} Linux 4.17.6 (in Fedora Release 28) and Nautilus @ commit \texttt{2fb4e52816}
  \item {\bf Hardware:} AMD EPYC 7281 16-Core Processor
  \item {\bf Compiler:} gcc 8.1.1 20180712 (Red Hat 8.1.1-5).
  \item {\bf Compiler Options:} See the \texttt{Makefile} and the \texttt{nautilus/Makefile} for compile-options.
  \item {\bf Experiment workflow:} See section.
  \item {\bf Experiment customization:} Modify what work the database is doing, the number of chunks, number of columns, chunk-size, and domain-size (domain of the elements in the database).
  \item {\bf Publicly available:} Yes
  \end{itemize}
}

\subsubsection{How software can be obtained (if available)}

NautDB can be cloned from this \href{https://github.com/HExSA-Lab/db-multiverse/}{GitHub Repository}.

\subsubsection{Hardware dependencies}

At the time of this writing, Nautilus supports x86\_64, Xeon Phi, and the GEM5 simulator.

In order to gather performance data, I use the CPU-enabled performance counters. These are specific within processor families. I have targetted `AMD EPYC 7281 16-Core Processor', but the code can be modified for other processors as well (see \texttt{src/app/perf*} and \texttt{include/app/perf*}). If this is not modified, you can still collect cycle-counts.

\subsubsection{Software dependencies}

gcc, GNU make, GNU libc (for running in Linux), grub2 (for compiling Nautilus), xorriso (for compiling Nautilus)

%%%%%%%%%%%%%%%%%%%%%%%%%%%%%%%%%%%%%%%%%%%%%%%%%%%%%%%%%%%%%%%%%%%%% 
\subsection{Installation}

\begin{enumerate}
\item Download the source code.

  \begin{shellcode}
    git clone git@github.com:\
    HExSA-Lab/db-multiverse.git
  \end{shellcode}
\end{enumerate}

%%%%%%%%%%%%%%%%%%%%%%%%%%%%%%%%%%%%%%%%%%%%%%%%%%%%%%%%%%%%%%%%%%%%% 
\subsection{Experiment workflow}

\subsubsection{Manual Workflow}

\begin{enumerate}[leftmargin=1cm]
\item Compile the Linux version
  \begin{shellcode}
    make main
  \end{shellcode}

\item Run and collect output
  \begin{shellcode}
    ./main > linux_output
  \end{shellcode}

\item Compile Nautilus
  \begin{shellcode}
    ./scripts/insert_into_nautilus.sh
    make -C nautilus nautilus.bin
  \end{shellcode}

\item Boot into Nautilus. If you are using grub, 
  \begin{shellcode}
    mv nautilus.bin /boot/nautilus.bin
    echo <<EOF
    menuentry "Nautilus" {
      # adjust for your specific hw
      set root='hd0,msdos1'
      multiboot2 /nautilus.bin
      boot
    }
    EOF >> /etc/grub.d/41_custom
    reboot
    # wait for grub menu
    # select Nautilus from the menu
    # capture output over serial link
  \end{shellcode}

\subsubsection{Workflow Automation}

This is the simplest way of reproducing the Nautilus experiment. However, the user can benefit from automation.

\begin{itemize}
\item A hardware management tool such as \texttt{IPMI} can automate the process of rebooting, selecting Nautilus from the Grub menu, and capturing the serial remotely. This is implemented in \texttt{scripts/ipmi\_helper.sh}.
\item \texttt{expect} can be used to automate navigating the Grub menu. This is implemented in \texttt{scripts/drive\_grub.py}.
\item See \texttt{scripts/run\_linux.sh} and \texttt{scripts/run\_nautk.sh} for start-to-finish automation on both platforms with a remote run-host and a remote build-host.
\end{itemize}

\end{enumerate}

%%%%%%%%%%%%%%%%%%%%%%%%%%%%%%%%%%%%%%%%%%%%%%%%%%%%%%%%%%%%%%%%%%%%% 
\subsection{Evaluation and expected result}

The software will output blocks of CSV data wrapped in curly-braces, such as:

\begin{shellcode}
  file: cool_data.csv {
    x column1 name,column2 name,title
    1,2,
  }
\end{shellcode}

The independent variables have a header beginning with `\texttt{x}'.

%%%%%%%%%%%%%%%%%%%%%%%%%%%%%%%%%%%%%%%%%%%%%%%%%%%%%%%%%%%%%%%%%%%%% 
\subsection{Experiment customization}

Customize \texttt{src/app/main.c} to choose which modules to run.

To customize the \texttt{test\_db} module, edit \texttt{src/app/test\_db.c}. The parameters for the database (number of columns, chunk size, number of chunks, domain size) and which operators will be timed can be customized here.

\subsection{Notes}
It may be tempting to run Nautilus in a virtualized environment such as QEMU instead of on bare-metal. While this can be an error-detecting step, but the virtualized environment will not have yield realistic performance data.

\end{document}

%%% Local Variables:
%%% mode: latex
%%% TeX-master: t
%%% End:
