\begin{block}{Discussion}

  \begin{itemize}
  \item Preliminary experiments demonstrate that:
    \begin{itemize}
    \item Specialization allows operations to be tuned for a workload/application
    \item Compiled queries that consist of highly-specialized code benefit from a hybrid runtime environment
    \item Hybrid runtimes have much more predictable performance than general purpose OS $\implies$ also will lead to better performance once we go parallel (e.g., bulk-synchronous model)
    \item Specialization significantly improves cache performance 
    \end{itemize}
  \item While Linux often outperforms our hybrid runtime, we can remedy this situation by introducing complementary specializations (as demonstrated in Fig.~\ref{fig:malloc}q) and choose them based on the workload
  \end{itemize}

  % \todo{
  %   Since we are specialized we can improve the performance for specific workloads.
  %   If we are not better, we could switch to different customizations (chosen by the application-programmer).
  % }

\end{block}
%%% Local Variables:
%%% mode: latex
%%% TeX-master: "main"
%%% End:
