\begin{block}{NautDB Prototype}
  \begin{itemize}

  \item Hand-optimized, in-memory, columnar, chunk-oriented operator
    implementations

  \item Tables are stored in a column-oriented fashion, the data of a
    columns is split into 256 chunks - each is an array of data values
    of a fixed size

  \item We tested creating, sorting, selection (filter), creating, and
    freeing tables

  \item The database server is compiled into Nautilus and invoked on
    startup (single-application OS)

    % I don't want to get into these details.
    % They are important for the paper, but not for the poster
    % The reviewers said we needed more detail on other things, so I
    % will trade the space.
    % \item We report \textbf{sorting} a relation consisting of 256 chunks:
    %   \begin{itemize}
    %   \item intra chunk: sort data using counting sort
    %   \item inter chunk: merge sort
    %   \item two sorting variants: row-oriented versus column-oriented
    %   \end{itemize}
    %   splitting the input into chunks, applying counting sort per chunk, and merge sort to sort across chunks
    % \item Basic array operations: allocate, set all, get all, copy, and free

    %   \begin{itemize}

    %     these parameters are detailed is duplicated in the graph.
    %     No need for them here.
    %   \item \textbf{\# of columns}: between 2 and 128
    %   \item \textbf{\# of chunks}: 256
    %   \item \textbf{chunk size (\# of rows per chunk)}: between 256 ($2^8$) and 16,384 ($2^{14}$)
    %   \end{itemize}

  \end{itemize}
\end{block}

\begin{block}{Configuration}
  \begin{itemize}
  \item We vary the \underline{number of columns} and the \underline{size of chunks}
  \item We aggregated over 10 trials
  \item Hardware: 16-core x86\_64 AMD EPYC 7281 with 4 NUMA nodes
  \item OS: linux kernel 4.17.6, Nautilus - git commit \texttt{2fb4e52816}
  \item Nautilus has default configuration with debugging removed and extra devices disabled. It uses 1GB pages.
  \end{itemize}
\end{block}

%%% Local Variables:
%%% mode: latex
%%% TeX-master: "main"
%%% End:
