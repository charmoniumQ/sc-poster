% https://www.overleaf.com/18225217jqvtscscccvs#/68889822/
% https://www.overleaf.com/18225348jztrfnwtzspz#/68890238/
% https://www.overleaf.com/read/hprgdrpcvvpm#/32910503/

%%%%%%%%%%%%%%%%%%%%%%%%%%%%%%%%%%%%%%%%% 
% Jacobs Portrait Poster
% LaTeX Template
% Version 1.0 (31/08/2015)
% (Based on Version 1.0 (29/03/13) of the landscape template
% 
% Created by:
% Computational Physics and Biophysics Group, Jacobs University
% https://teamwork.jacobs-university.de:8443/confluence/display/CoPandBiG/LaTeX+Poster
% 
% Further modified by:
% Nathaniel Johnston (nathaniel@njohnston.ca)
% 
% Portrait version by:
% John Hammersley
% 
% The landscape version of this template was downloaded from:
% http://www.LaTeXTemplates.com
% 
% License:
% CC BY-NC-SA 3.0 (http://creativecommons.org/licenses/by-nc-sa/3.0/)
% 
%%%%%%%%%%%%%%%%%%%%%%%%%%%%%%%%%%%%%%%%% 
% DOC CLASS AND PACKAGES
\documentclass[final,14pt]{beamer}
\usepackage{graphicx}
\usepackage{enumitem}
\usepackage[framemethod=tikz]{mdframed}
\setbeamerfont{caption}{family=\rmfamily, size=\normalsize}
%\setbeamerfont{normal text}{family=\rmfamily}%, size=\small}

%%%%%%%%%%%%%%%%%%%%%%%%%%%%%%%%%%%%%%%%
% Use the confposter theme supplied with this template
\usetheme{confposter}

%%%%%%%%%%%%%%%%%%%%%%%%%%%%%%%%%%%%%%%%
% HEADER
%%%%%%%%%%%%%%%%%%%%%%%%%%%%%%%%%%%%%%%%%%%%%%%%%%%%%%%%%%%%%%%%%%%%%%%%%%%%%%%%%%%%%%%%%%%%%%%%%%%%
\setbeamertemplate{headline}{
  \leavevmode

  \begin{beamercolorbox}[wd=\paperwidth]{headline}
    \begin{columns}[T]
      \begin{column}{.02\paperwidth}
      \end{column}
      \begin{column}{.16\paperwidth}
        \includegraphics[width=0.05\paperwidth]{dbgrouplogo.pdf}\hspace{0.005\paperwidth}%
        \includegraphics[width=0.07\paperwidth]{hexsalablogo.png}%
      \end{column}
      \begin{column}{.64\paperwidth}
        \vskip4ex
        \centering
        \usebeamercolor{title in headline}{\color{fg}\textbf{\huge{\inserttitle}}\\[1ex]}
        \usebeamercolor{author in headline}{\color{fg}\Large{\insertauthor}\\[1ex]}
        \usebeamercolor{institute in headline}{\color{fg}\large{\insertinstitute}\\[1ex]}
      \end{column}
      \begin{column}{.18\paperwidth}
        \vskip2cm
        \begin{center}
          \raggedright
          \includegraphics[width=.08\paperwidth]{utd_logo.jpg}\hspace{0.005\paperwidth}%
          \includegraphics[width=.08\paperwidth]{iit_logo.png}%
        \end{center}
        \vskip1.5cm
      \end{column}
      % \begin{column}{.02\paperwidth}
      % \end{column}
    \end{columns}
  \end{beamercolorbox}

   \vspace{0.5in}
 \hspace{0.5in}\begin{beamercolorbox}[wd=1\linewidth,colsep=0.15cm]{cboxb}\end{beamercolorbox}
}

%%%%%%%%%%%%%%%%%%%%%%%%%%%%%%%%%%%%%%%%
% COLORS
%%%%%%%%%%%%%%%%%%%%%%%%%%%%%%%%%%%%%%%%
\definecolor{mylinecolor}{rgb}{0.122, 0.435, 0.698}
% Many more colors are available for use in beamerthemeconfposter.sty
% \setbeamercolor{block title}{fg=ngreen,bg=white}
% \setbeamercolor{block body}{fg=black,bg=white}
% \setbeamercolor{block alerted title}{fg=white,bg=dblue!70}
% \setbeamercolor{block alerted body}{fg=black,bg=dblue!10}

% \addtobeamertemplate{block begin}{\pgfsetfillopacity{0.0}}{\pgfsetfillopacity{1}}

% \usebackgroundtemplate{
% \begin{tikzpicture}
%   \path [outer color = green!10, inner color = blue!10]
%   (0,0) rectangle (\paperwidth,\paperheight);
% \end{tikzpicture}}

%%%%%%%%%%%%%%%%%%%%%%%%%%%%%%%%%%%%%%%%
% LAYOUT
%%%%%%%%%%%%%%%%%%%%%%%%%%%%%%%%%%%%%%%%

% Use the beamerposter package for laying out the poster (cm)
% https://github.com/deselaers/latex-beamerposter/blob/master/beamerposter.sty#L138
\usepackage[scale=1.85, size=custom, width=240, height=120]{beamerposter}

% Define the column widths and overall poster size

% To set effective sepwid, onecolwid and twocolwid values, first choose how many columns you want and how much separation you want between columns
% In this template, the separation width chosen is 0.024 of the paper width and a 4-column layout
% onecolwid should therefore be (1-(# of columns+1)*sepwid)/# of columns e.g. (1-(4+1)*0.024)/4 = 0.22
% Set twocolwid to be (2*onecolwid)+sepwid = 0.464
% Set threecolwid to be (3*onecolwid)+2*sepwid = 0.708
\newlength{\sepwid}
\newlength{\onecolwid}
\setlength{\paperwidth}{240cm}
\setlength{\paperheight}{120cm}
\setlength{\sepwid}{0.01\paperwidth} % Separation width (white space) between columns
\setlength{\onecolwid}{0.24\paperwidth} % Width of one column
\setlength{\topmargin}{-1in} % Reduce the top margin size

\addtobeamertemplate{block end}{}{\vspace*{2ex}} % White space under blocks
\addtobeamertemplate{block alerted end}{}{\vspace*{2ex}} % White space under highlighted (alert) blocks

% \setlength{\belowcaptionskip}{2ex} % White space under figures
% \setlength\belowdisplayshortskip{2ex} % White space under equations

\newcommand{\samelineand}{\qquad}

% http://mirror.hmc.edu/ctan/macros/latex/contrib/enumitem/enumitem.pdf
\setlist[itemize]{label=\textbullet,leftmargin=1.5cm}
\setlist[itemize,1]{itemsep=0.8cm,itemsep=0.8cm}
\setlist[itemize,2]{itemsep=0.2cm,itemsep=0.5cm}

\newcommand{\todo}[1]{\textbf{\textcolor{red}{TODO: #1}}}
\newcommand{\buzzword}[1]{\textbf{#1}}
\newcommand{\insight}[1]{\alert{#1}}

\title{\emph{NautDB}: Towards a Hybrid Runtime for Processing Compiled Queries}
\author{Samuel Grayson (sag150430@utdallas.edu)}
\institute{University of Texas at Dallas}

\graphicspath{{./imgs/}}


%%%%%%%%%%%%%%%%%%%%%%%%%%%%%%%%%%%%%%%%
% OVERVIEW BOX
\newmdenv[linewidth=10pt,innerlinewidth=0.5pt, roundcorner=40pt,linecolor=dblue,innerleftmargin=26pt,
innerrightmargin=26pt,innertopmargin=26pt,innerbottommargin=26pt]{overviewbox}


% ----------------------------------------------------------------------------------------
\begin{document}

\begin{frame}[t] % The whole poster is enclosed in one beamer frame

  \begin{columns}[t]

    \begin{column}{\sepwid}
    \end{column}

    \begin{column}{\onecolwid}
      


\begin{block}{Abstract}
  \begin{itemize}
  \item   With the explosion of ``big data'', executing queries on vast amounts of data in parallel has become a bottleneck in many HPC systems and beyond.
  \item   Complex multi-layered abstractions behind generic interfaces in the traditional OS and database software stack are getting in the way of exploiting the characteristics of modern multi-core systems to maximize performance.
  \item   To mitigate this inefficiency, we built \textbf{NautDB}, a \textbf{hybrid runtime kernel} (based on the \textbf{Nautilus Aerokernel}~\cite{HALE:2015:NAUTILUS}) for the parallel execution of \textbf{compiled queries}, thus, combining for the first time the concepts of hybrid runtimes and compiled query processing developed by the operating system and database communities.
  \item   We implement a \textbf{testing prototype} to evaluate the potential performance benefits that can be achieved by our specialized hybrid runtime against a general-purpose runtime, for which Linux will serve as our exemplar.
  \end{itemize}
\begin{overviewbox}
    \textbf{\emph{The goal of this research is to integrate specialization techniques from the OS community
      (unikernels) and  DB community (compiled queries) to achieve high-performance.}}
\end{overviewbox}
\end{block}

%%% Local Variables:
%%% mode: latex
%%% TeX-master: "main"
%%% End:

      % \vspace{-2cm}
\begin{block}{Status quo}
  \begin{itemize}
  \item {General purpose OSs}
    \begin{itemize}
     \item Too many abstractions over HW $\implies$ high-performance is difficult
      to achieve~\cite{GICEVA:2016:OS_SUPPORT,HALE:2015:NAUTILUS}
    \item Not workload aware $\implies$ optimized for the general case
      \begin{itemize}
      \item e.g. memory allocation scheme
      \end{itemize}
    \item Time-sharing kernel $\implies$ context-switches, scheduling interrupts, and thread-migration
    \item Separation of kernel and user space $\implies$ more context switches
    \end{itemize}
%      use specialized scheduling and memory management abstractions.
 %   \item DBs suffer from over lack of specialization
  \item {General purpose DB}
    \begin{itemize}
    \item Interpreted query and expression evaluation
      \begin{itemize}
      \item Deep function-call trees
      \item Poor spacial locality in instruction memory
      \end{itemize}
    \end{itemize}
  \end{itemize}
\end{block}
%%% Local Variables:
%%% mode: latex
%%% TeX-master: "main"
%%% End:

      \begin{block}{Specialized Hybrid Runtimes~\cite{HALE:2015:NAUTILUS}}
  \begin{itemize}
  \item Kernel + Runtime run in ring 0 (unikernel-like) $\implies$ fewer context switches
  \item Partition physical resources between general purpose OS and hybrid runtime~\cite{KOCOLOSKI:2015:PISCES} $\implies$ can call general purpose OS where needed.

    % Advances in multi-kernel environments allow physical \todo{(cite pisces)} resources to be split between a full-weight kernel and a hybrid runtime.
    % \item \alert{Unikernel-inspired design prevents the OS from `getting in the way' of the app-programmer.}
  \item {Unikernel-inspired design gives programmer fine-grained control over \ldots}
    \begin{itemize}
    \item Not time-shared $\implies$ No context-switches or thread-migration
    \item Avoid interrupts $\implies$ Faster and more predictable
    \item Single address space $\implies$ Huge page-sizes (1Gb) $\implies$ No TLB misses
    \item Memory allocation $\implies$ Specialize allocator for workload/application (see Fig. \ref{fig:malloc})
    \end{itemize}
  \end{itemize}
\end{block}

%%% Local Variables:
%%% mode: latex
%%% TeX-master: "main"
%%% End:

      
\begin{block}{Compiled Query Processing~\cite{SK16,N11}}
  \begin{itemize}
    \item DB engine is specialized for a particular query
    \item Can be pre-compiled or JIT-compiled
      \begin{itemize}
      \item Smaller code $\implies$ less i-cache misses and branches
      \item Specialized data structures
      \item Combine code for multiple operators (pipeline) $\implies$ optimizations within operators and less func calls
      \item Specialize code for hardware without blow up in code size
      \end{itemize}
    \item The DB-programmer writes compiled code in the unikernel, giving them unfettered access HW.
      \begin{itemize}
      \item Compose operators within a block $\implies$ avoid function-call overhead.
      \item Control HW $\implies$ NUMA-aware code.
      \end{itemize}
  \end{itemize}
\end{block}

%%% Local Variables:
%%% mode: latex
%%% TeX-master: "main"
%%% End:

    \end{column}

    \begin{column}{\sepwid}
    \end{column}

    \begin{column}{\onecolwid}
      \begin{figure}
  \vspace{-1cm}
  \includegraphics[height=34cm]{plots/malloc.png}
  \vspace{-1cm}
  \label{fig:malloc}
  \caption{~The custom allocator outperforms the highly optimized allocator of linux for smaller sizes by more than an order of magnitude. This demonstrates the potential of specialization---the behaviour of a component can be fine-tuned to workload characteristics.}
\end{figure}

%%% Local Variables:
%%% mode: latex
%%% TeX-master: "main"
%%% End:

      \begin{block}{Hypothesis}
\begin{overviewbox}
    \textbf{Executing compiled queries in a hybrid runtime can improve performance within parameter ranges that benefit from a particular specialization and leads to more predictable performance.}
\end{overviewbox}
  
\end{block}

%%% Local Variables:
%%% mode: latex
%%% TeX-master: "main"
%%% End:

      % \begin{block}{Previous Work}
  \begin{itemize}
  \item Customized OS support for data-processing: \todo{I am not sure what specific improvements we have over this paper.}
  \item A Case for Transforming Parallel Runtimes Into Operating System Kernels: the authors use a specialized runtime and show performance benefit for runtimes such as Legion.
  \item System-Level Support for Composition of Applications: the authors introduce Pisces which allows multiple kernels to run on the same physical node, partitioning the resources. This makes it more convenient to run an single-application OS alongside a general-purpose OS for other applicatoins.
  \item \todo{get a paper on Compiled Queries}
  \end{itemize}
  
\todo{Cite these from the bibliogrpahy}
\end{block}
%%% Local Variables:
%%% mode: latex
%%% TeX-master: "main"
%%% End:

      \begin{block}{Testing Prototype and Experimental Setup}
  \begin{itemize}
    \item \textbf{Workload}
      \begin{itemize}
      \item Hand-optimized, in-memory, columnar, chunk-oriented operator
        implementations
    \begin{itemize}
    \item sorting, selection (filter), creating, and freeing tables
    \end{itemize}
  \item Basic array operations: allocate, set all, get all, copy, and free
  \end{itemize}
   \item \textbf{Compared Configurations}
    \begin{itemize}
    \item Hybrid runtime using Nautilus~\cite{HALE:2015:NAUTILUS} versus as application in Linux 
    \end{itemize}
  \item \textbf{Setup}
    \begin{itemize}
    \item hardware: 16-core x86\_64 AMD EPYC with 4 NUMA nodes
    \item OS: linux kernel 4.17.6, Nautilus \verb+@9df0e062+
    \end{itemize}
  \end{itemize}
\end{block}
%%% Local Variables:
%%% mode: latex
%%% TeX-master: "main"
%%% End:

    \end{column}

    \begin{column}{\sepwid}
    \end{column}

    \begin{column}{\onecolwid}
      \vspace{-1cm}
      \begin{block}{Experimental Results}
 
  \begin{figure}
    \includegraphics[height=30cm]{plots/sort_2d.png}
    \todo{change log axis labels to non-log, e.g., $2^3 = 8$ and samce for the legend ($10^x$ to the actual factor).}
    \caption{~Row-oriented sort in Nautilus and Linux varying the \#columns and chunk-size}
    \label{fig:sort_2d}
  \end{figure}

  For larger number of columns and chunk sizes, Nautilus outperforms Linux. This is because Nautilus has larger page size and incurs less TLB misses (see Table \ref{table:cache_miss}).

    \begin{table}
      \bgroup
      \def\arraystretch{1.3}%
      \setlength\tabcolsep{1cm}
      \begin{tabular}{l || r | r }
        \textbf{Metric}  & \textbf{Linux} & \textbf{Nautilus} \\
        \hline\hline
        TLB misses               & \todo{} & \todo{} \\
        Instruction cache misses & \todo{} & \todo{}  \\

        % I don't have perf-counter data for the following
        % (there is no perf counter)
%        Page faults              &     & 0 \\
%        Context switches         &     & 0 \\
%        Interrupts               &     & 0 \\
      \end{tabular}
\egroup
      \label{table:cache_miss}
      \caption{~Row-oriented sorting for a $128$ column table with 256 chunks with $8192$ elements each}
    \end{table}
  
  \begin{figure}
    \includegraphics[height=30cm]{plots/sort.png}
    \todo{for plot: change log cols to cols, increase font size}
    \caption{~Column-oriented sort measuring runtime and uncertainty (\todo{2 standard deviations?}) for a fixed chunk size \todo{WHICH ONE}, varying the number of columns.}
    \label{fig:sort}
  \end{figure}

  Furthermore, we see that Nautilus performance if much more predictable than Linux. This effect is observed even in configurations where Linux outperforms Nautilus. The main reason for this is that Nautilus does not have scheduling interrupts, so it avoids unpredictable detours which also leads to better cache performance (see Table \ref{table:cache_miss-col}).

    \begin{table}
      \bgroup
      \def\arraystretch{1.3}%
      \setlength\tabcolsep{1cm}
      \begin{tabular}{l || r | r }
        \textbf{Metric}  & \textbf{Linux} & \textbf{Nautilus} \\
        \hline\hline
        TLB misses               & 8,000,000 & 1,000,000 \\
        Instruction cache misses & 2,000,000 & 0  \\

        % I don't have perf-counter data for the following
        % (there is no perf counter)
%        Page faults              &     & 0 \\
%        Context switches         &     & 0 \\
%        Interrupts               &     & 0 \\
      \end{tabular}
\egroup
      \caption{~Column-oriented sorting for a $128$ column table with 256 chunks with $8192$ elements each}
      \label{table:cache_miss-col}
    \end{table}
  

\end{block}

%%% Local Variables:
%%% mode: latex
%%% TeX-master: "main"
%%% End:

    \end{column}

    \begin{column}{\sepwid}
    \end{column}

    \begin{column}{\onecolwid}
      \begin{block}{Discussion}
  \alert{\textbf{TODO: discuss}}
\end{block}
%%% Local Variables:
%%% mode: latex
%%% TeX-master: "main"
%%% End:

      %\begin{block}{Conclusion}
  The research is promising, but more work needs to be done. Future work should seek to understand and improve the preformance details such as TLB misses and data cache misses in order to beat a general purpose OS.
\end{block}
%%% Local Variables:
%%% mode: latex
%%% TeX-master: "main"
%%% End:

      \begin{block}{Next Steps}

  \begin{itemize}
  \item Borrow algorithms from Linux where they outperform the existing Nautilus algorithms
  \item Make the development process simpler for the user
  \item Evaluate parallel implementations of operators
  \item Evaluate effect of noise due to other applications running (in a partitioned VM environment)
  \end{itemize}

\end{block}

%%% Local Variables:
%%% mode: latex
%%% TeX-master: "main"
%%% End:

      \begin{block}{Our Vision: \textbf{NautDB}}
  \begin{itemize}
  \item \textbf{NautDB}: Implement our vision of \textit{NautDB} as a
    hybrid dataflow engine for compiled queries
  \end{itemize}
\end{block}



%%% Local Variables:
%%% mode: latex
%%% TeX-master: "main"
%%% End:

      \begin{block}{References}
{\scriptsize
  \bibliographystyle{abbrv}
  \bibliography{kyle,sam,boris}
  }
\end{block}

%%% Local Variables:
%%% mode: latex
%%% TeX-master: "main"
%%% End:

    \end{column}

    \begin{column}{\sepwid}
    \end{column}
  \end{columns}
  
\end{frame}

\end{document}

%%% Local Variables:
%%% mode: latex
%%% TeX-master: t
%%% End:
